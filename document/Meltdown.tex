\section{Meltdown}\label{sec:meltdown}
Meltdown wurde zusammen mit Spectre von derselben Forschergruppe am 3.~Januar 2018 veröffentlicht~\cite{kocher2018spectre}.
Im Gegensatz zu Spectre ist Meltdown ein Angriff, der primär auf Intel-Prozessoren abzielt, wurde jedoch auch auf der PowerPC-Architektur und einigen ARM-Prozessoren nachgewiesen~\cite{cve_2024_45333}.
Das Funktionsprinzip von Meltdown ähnelt dem von Spectre, da beide Angriffe den Cache als Side-Channel nutzen, um unzugängliche Daten zu extrahieren.
Der wesentliche Unterschied zu Spectre liegt jedoch in der Art und Weise, wie die Befehle ausgeführt werden, die auf diese Daten zugreifen.
Während Spectre auf spekulative Ausführung setzt, nutzt Meltdown die Out-of-Order-Ausführung aus~\cite{kocher2018meltdown}.

\subsection{Funktionsweise}
\label{subsec:meltdown_funktionsweise}
Um Kontextwechsel zwischen Benutzer- und Kernelmodus zu beschleunigen wurde auf Betriebssystemebene der Speicher des Kernels und der Benutzerprozesse im selben Adressraum verfügbar gemacht, anstatt wie bei Prozessen üblich, voneinander getrennt zu sein.
Um zu verhindern, dass ein Benutzerprozess auf den Speicher des Kernels zugreift, wird beim Wechsel in den Kernelmodus ein spezielles Bit gesetzt, das steuert, ob Zugriff auf den Kernel-Speicher erlaubt ist.
Dieses Bit kann im Benutzerprozess nicht verändert werden und wird automatisch beim Wechsel in den Usermodus wieder entfernt.
Dadurch entfällt bei Kontextwechseln die erneute Konfiguration der Memory Management Unit (MMU), da der Kernel-Speicher bereits im Adressraum verfügbar ist.
Sollte ein Benutzerprozess dennoch versuchen, auf den Kernel-Speicher zuzugreifen, wird dieser Zugriff vom Betriebssystem abgefangen und führt zu einer Segmentation Fault.
Somit sollte es eigentlich nicht möglich sein, Daten des Kernels zu laden oder diese für weitere Verarbeitung zu nutzen. \\
Durch Hardwarefehler ist es möglich, dass durch Out-of-Order-Ausführung das Laden von Daten aus dem Kernel-Speicher abgeschlossen wird, bevor der Zugriff auf den Kernel-Speicher überprüft wurde.
Das Verfahren, um einen Wert aus dem Cache auslesbar zu machen, nutzt die bereits bei Spectre beschriebene Technik: Basierend auf dem geladenen Wert wird ein Teil eines Arrays geladen, und anschließend werden die Zugriffszeiten auf alle Bereiche des Arrays gemessen, um den geladenen Bereich zu ermitteln.
Dies lässt Rückschlüsse auf den ursprünglich geladenen Wert zu, obwohl dieser Zugriff illegal war und der Prozessor den Zugriff rückgängig gemacht hat. \\
Ein Beispiel in C-Code könnte wie folgt aussehen:
\texttt{array2[*kernel\_addr * 512]}\\
Der Stern vor \texttt{kernel\_addr} dereferenziert den Pointer in den Kernelspeicher und lädt den Wert, der sich an dieser Adresse befindet.
Dieser wird anschließend mit \texttt{512} multipliziert, um einen bestimmten Bereich des Arrays \texttt{array2} zu laden.
Im Gegensatz zu Spectre wird dieser Code nicht spekulativ ausgeführt.
Stattdessen verursacht er eine Segmentation Fault, da die Prüfung des Zugriffs auf den Kernel-Speicher durch die Out-of-Order-Ausführung weiterhin stattfindet, jedoch verzögert.\\
Um zu verhindern, dass das Programm beim Auslösen der Segmentation Fault abstürzt, sind verschiedene Techniken möglich:
Einerseits kann eine geworfene Segmentation Fault durch Exception Handling abgefangen werden, sodass der Programmfluss an einer anderen Stelle fortgesetzt werden kann.
Alternativ kann der Prozess kurz vor dem Zugriff geforkt werden, wodurch der Zugriff auf den Kernel-Speicher in einem separaten Prozess erfolgt, welcher daraufhin abstürzt, während der ursprüngliche Prozess die Zeitmessung durchführt.
\subsection{Gegenmaßnahmen}
\label{subsec:meltdown_gegenmassnahmen}
Da wie bei Spectre sowohl Hardwarehersteller als auch Softwareentwickler bereits einige Zeit vor der Veröffentlichung von Meltdown informiert wurden, konnten vorab verschiedene Gegenmaßnahmen implementiert werden.
Die bekannteste Gegenmaßnahme ist die Einführung von Kernel Page Table Isolation (KPTI, ehemals KAISER), die den Kernel-Speicher in einen separaten Adressraum auslagert.
Bereits im Jahr 2014 wurde KASLR (Kernel Address Space Layout Randomization) im Linux-Kernel implementiert, um Kernel-Adressen vor Benutzerprozessen zu verbergen.
Der Nachfolger KAISER (Kernel Address Isolation to have Side-channels Efficiently Removed) bot zusätzlich die Möglichkeit, den Kernel-Speicher vor Benutzerprozessen zu schützen und den Meltdown-Angriff zu verhindern.
KPTI erweitert KAISER, indem es den Kernel-Speicher vollständig in einen separaten Adressraum auslagert und somit verhindert, dass Benutzerprozesse auf diesen zugreifen können.
Die Performance-Einbußen durch KPTI variieren je nach Anwendungsfall und liegen zwischen 5\% bei \enquote{regulären} Anwendungen und bis zu 30\% in extremen Fällen~\cite{kaiser_2017}.
Intel konnte die Performance-Einbußen jedoch durch Hardware-Updates reduzieren, indem Prozessoren mit PCID (Process-Context Identifiers) ausgestattet wurden~\cite{pcworld_2018}.