\section{Spectre}
\label{sec:spectre}
Spectre wurde von verschiedenen Forschenden entdeckt und am 3.~Januar~2018 veröffentlicht~\cite{kocher2018spectre}.
Hardwarehersteller wie Intel und AMD sowie Softwarehersteller wie Microsoft wurden bereits am 1.~Juni~2017 informiert, um die Sicherheitslücke im Voraus durch Softwareupdates zu schließen. \\
Spectre ist ein Angriff, der Daten über einen Cache-Side-Channel ausliest.
Side-Channels sind indirekte Informationen, die über ein System gewonnen werden können, wie beispielsweise der Stromverbrauch der CPU, elektromagnetische Strahlung, Zeitmessungen oder sogar Geräusche.
Spectre analysiert die Zugriffszeiten auf Daten, um zu bestimmen, ob diese aus dem Arbeitsspeicher geladen werden mussten, was mehrere Hundert Zyklen dauert, oder ob sie durch vorherige Zugriffe bereits im Cache vorhanden sind, was weniger als zehn Zyklen benötigt.
Um diese Daten in den Cache zu laden, nutzt Spectre die spekulative Ausführung von Befehlen, um Daten zu laden, die eigentlich nicht geladen werden dürften.
Die spekulative Ausführung ermöglicht das Laden dieser Daten, da die Überprüfung, ob der Zugriff auf diese Daten überhaupt legitim ist, erst bei der \enquote{tatsächlichen} und nicht spekulativen Ausführung erfolgt.

\subsection{Vorgehensweise}
\label{subsec:spectre_vorgehensweise}
\begin{itemize}
	\item[\textbf{1.}] 2 Arrays mit verschiedenen Größen (\texttt{array1: 16 Elemente, array2: 265 * 512 Elemente}) werden allokiert, die später für spekulative Zugriffe und Cache-Messungen verwendet werden.
	\item[\textbf{2.}] Es wird berechnet, wie weit der auszulesende Speicherbereich vom Anfang von \texttt{array1} entfernt ist.
	      Dieser Abstand wird benötigt, um einen Index zu bestimmen, der beim Zugriff auf \texttt{array1} tatsächlich auf den Zielbereich verweist (Im nachfolgenden Beispiel \texttt{31000}).
\end{itemize}

\noindent
\begin{minipage}{0.58\textwidth}
	\begin{itemize}
		\item[\textbf{3.}] Spectre nutzt folgenden Code, um den Cache zu trainieren: \\
		      \phantomsection\label{itm:spectre_step3_code}
		      \texttt{array2[array1[x] * 512]} \\
		      Der Zugriff verwendet den Wert, der sich an der Position \texttt{x} in \texttt{array1} befindet, multipliziert diesen mit 512 und greift anschließend auf das entsprechende Element in \texttt{array2} zu.
		      Ein solcher Aufruf mit dem Wert \texttt{x=2} hätte zur Folge, dass auf den Wert an der Stelle \texttt{array2[array[2] * 512]} zugegriffen wird.
		      Befindet sich an der Stelle \texttt{array1[2]} der Wert \texttt{42}, wird so der Teil des \texttt{arrays2} and der Stelle \texttt{array2[42 * 512]} in den Cache geladen.
	\end{itemize}
\end{minipage}
\hfill
\begin{minipage}{0.4\textwidth}
	\centering
	\includegraphics[width=\linewidth]{Attack}
\end{minipage}

\begin{itemize}
	\item[\textbf{4.}]\phantomsection\label{itm:spectre_step4} Die oben gezeigte Zeile wird mehrfach in einer Schleife ausgeführt, um den Branch Predictor zu trainieren.
	      Dabei werden \texttt{x}-Werte verwendet, die innerhalb des \texttt{array1} liegen.
	      Dies führt dazu, dass die CPU annimmt, zukünftige Durchläufe der Schleife würden denselben Aufruf ausführen,
	      was zur Folge hat, dass dieser für zukünftige \texttt{x} Werte bereits im Voraus spekulativ ausgeführt wird.
	\item[\textbf{5.}] Der Wert \texttt{x} wird nun auf einen Index gesetzt, der außerhalb des \texttt{array1} liegt.
	      Dieser Index entspricht dem Abstand zwischen \texttt{array1} und einem Speicherbereich, der durch den Exploit ausgelesen werden soll.
	      Durch das vorherige Training führt die CPU einen spekulativen Zugriff auf \texttt{array1} aus, obwohl der Wert \texttt{x} außerhalb des \texttt{array1} liegt.
	      Beispielsweise könnte \texttt{x=31000} gesetzt werden, was dazu führt, dass die CPU spekulativ auf die Adresse \texttt{array1 + 31000} zugreift.
	      Da \texttt{array1} jedoch nur 16 Elemente enthält, liegt diese Adresse außerhalb des gültigen Adressbereichs und verweist stattdessen auf einen anderen Speicherbereich.
	      Der Wert des Bytes an dieser Adresse wird mit 512 multipliziert und verwendet, um auf ein Element in \texttt{array2} zuzugreifen.
	      Dieser Zugriff hat keinen direkten Nutzen, da \texttt{array2} keine relevanten Daten enthält.
	      Er bewirkt jedoch, dass der entsprechende Wert sowie einige nachfolgende Elemente von \texttt{array2} in den Cache geladen werden.
	      Die Abstände von 512 Bytes zwischen den Elementen in \texttt{array2} sorgen dafür, dass nur ein einzelner Wert in den Cache geladen wird.
	\item[\textbf{6.}] Eine spezielle Abbruchbedingung prüft, ob \texttt{x} innerhalb des \texttt{array1} liegt, und verhindert so, dass der Zugriff auf \texttt{array1[x]} tatsächlich ausgeführt wird, um einen Segmentation Fault zu vermeiden.
	      Dadurch wird die spekulative Ausführung des Zugriffs auf \texttt{array1} verworfen, sodass der Wert an dieser Stelle im weiteren Verlauf des Programms nicht mehr verwendet werden kann.
	      Die spekulative Ausführung ignoriert jedoch diese Abbruchbedingung, da sie in vorherigen Durchläufen bereits zu keinem Abbruch geführt hat.
	\item[\textbf{7.}] Nun werden alle 255 möglichen Werte, die sich an der Stelle \texttt{array1[x]} befinden könnten, nacheinander getestet.
	      Hierfür wird in einer Schleife gemessen, wie lange ein Zugriff auf das Element \texttt{array2[i * 512]} dauert.
	      Falls sich an der Stelle \texttt{array1[x]} z.B.~der Wert \texttt{4} befunden hat, würde dies bedeuten, dass die spekulative Ausführung das Element \texttt{array2[4 * 512]} in den Cache geladen hat.
	      Wird nun gemessen, dass der Zugriff auf \texttt{array2[4 * 512]} deutlich schneller ist als der Zugriff auf andere Elemente wie \texttt{array2[0 * 512]} oder \texttt{array2[5 * 512]}, so kann davon ausgegangen werden, dass sich an der Stelle \texttt{array1[x]} der Wert \texttt{4} befunden hat.
	\item[\textbf{8.}] Der ermittelte Wert wird gespeichert, und der \texttt{x}-Wert um 1 Byte inkrementiert.
	      Dadurch wird im nächsten Durchlauf das nächste Byte des Zielbereichs ausgelesen.
	      Falls noch nicht alle gewünschten Bytes ausgelesen wurden, wird zu \hyperref[itm:spectre_step4]{\underline{Schritt 4}} zurückgekehrt.
\end{itemize}

\subsection{Spectre-V2}
\label{subsec:spectre_spectre-v2}
Die oben beschriebene Variante von Spectre ist nur eine von vielen verschiedenen Varianten und Weiterentwicklungen.
Die bekannteste Weiterentwicklung ist Spectre-V2~\cite{spec_v2}, die gezielt den Branch Predictor angreift.
Seit der ersten Version von Spectre-V2 im Jahr 2018 (SpectreBTB) wurden kontinuierlich neue Varianten und Abwandlungen entwickelt, wie z.B.~Retbleed~\cite{spec_v2_retbleed} aus dem Jahr 2022 oder Training~Solo~\cite{training_solo} aus dem Jahr 2025. \\
Wie bei Spectre-V1 wird spekulativ Code ausgeführt, der Daten in den Cache lädt, die anschließend über Timing-Angriffe ausgelesen werden können.
Anstatt jedoch auf die spekulative Ausführung im eigenen Programm zu setzen, nutzt SpectreBTB aus, dass der Branch Prediction Buffer (BPB) zwischen Prozessen geteilt wird.
Dieser Buffer beinhaltet Daten, die für die Vorhersage von Branches verwendet werden und während der Programmausführung gesammelt werden.
Wenn die CPU im Programmablauf auf eine Branch trifft, deren Sprungziel noch unklar ist, wird der BPB verwendet, um den wahrscheinlichsten Sprung zu bestimmen und die spekulative Ausführung zu ermöglichen. \\
Um einen Angriff auszuführen, wird der BPB in unserem Programm so trainiert, dass er zukünftig Branches vorhersagt, die für uns vorteilhaft sind.
Wenn die CPU anschließend ein anderes Programm ausführt, in dem sich eine Branch mit unbekanntem Sprungziel befindet, wird der BPB verwendet, um den wahrscheinlichsten Sprung zu bestimmen.
Durch unser Training sagt der Branch Predictor nun einen Sprung voraus, der im Zielprogramm tatsächlich nie ausgeführt worden wäre.
Dies ermöglicht es uns, wenn auch nur spekulativ, den Ablauf eines anderen Programms zu beeinflussen. \\
Um Daten in den Cache zu laden, muss zunächst das Zielprogramm analysiert werden, um ein sogenanntes Gadget zu identifizieren.
Ein Gadget ist ein spezifischer Codeabschnitt, der für den Angriff genutzt werden kann, da er eine bestimmte Funktionalität bereitstellt, die für die spekulative Ausführung ausgenutzt werden kann.
Dieser Codeabschnitt hat eine spezifische Adresse, die durch Analyse des Programms ermittelt werden muss.
In unserem Fall handelt es sich um einen Codeabschnitt, der dem Zugriff \texttt{array2[array1[x] * 512]} aus \hyperref[itm:spectre_step3_code]{\underline{Schritt 3}} der Spectre-V1-Attacke ähnelt.
Nach der Identifikation dieses Gadgets wird dessen Adresse verwendet, um den Branch Prediction Buffer (BPB) zu trainieren. \\
Das Zielprogramm führt anschließend spekulativ das Gadget aus, wodurch Daten in den Cache geladen werden, die durch Zeitmessungen ausgelesen werden können.

\subsection{Gegenmaßnahmen}
\label{subsec:spectre_gegenmassnahmen}
Gegenmaßnahmen gegen Spectre-V2 wurden auf verschiedenen Ebenen umgesetzt, darunter zahlreiche Software-Patches.
Im Linux-Kernel wurden verschiedene Maßnahmen eingeführt, um zu verhindern, dass sensitive Funktionen spekulativ ausgeführt werden~\cite{linux_nospec}.
Dies verhindert, dass der Branch Predictor durch Manipulation innerhalb des Kernels Daten über spekulative Ausführung leaken kann.
Die Funktion \texttt{array\_index\_nospec} kann verwendet werden, um trotz spekulativer Ausführung Zugriffe auf die Bereiche von Arrays zu begrenzen.
Diese Funktion muss jedoch manuell in den Code eingefügt werden und ist nicht automatisiert implementiert.
Darüber hinaus wurden verschiedene Maßnahmen in Browsern implementiert, um Spectre-V2-Angriffe über JavaScript zu verhindern.
In Firefox und Webkit wurde die Genauigkeit von Zeitmessungen reduziert, um zu verhindern, dass Angreifer den Cache-Zugriff über Timing-Angriffe auslesen können (\texttt{performance.now()} wurde auf eine Genauigkeit von 20µs reduziert~\cite{luke_wagner_2018}).
Chrome hat die Sandbox-Funktionalität erweitert, sodass Websites in separaten Prozessen ausgeführt werden können~\cite{heise_2018_spec}.
Des Weiteren haben Hardwarehersteller wie Intel und AMD Microcode-Updates veröffentlicht, um Sicherheitslücken wie Spectre und Meltdown zu schließen. \\
Microcode ist eine spezielle, niedrigstufige Software, die direkt auf der CPU ausgeführt wird und Assembly-Befehle sowie komplexe Instruktionen in einfache Hardware-Steuersignale übersetzt.
Er ermöglicht es, bestimmte Funktionen der Hardware zu ändern oder zu erweitern, ohne physische Änderungen an der Hardware vorzunehmen. \\
Im Fall von Intel wurden Maßnahmen wie IBRS (Indirection Branch Restricted Speculation~\cite{intel_1}), IBPB (Indirect Branch Predictor Barrier~\cite{intel_2}) und STIBP (Single Thread Indirect Branch Predictors~\cite{intel_3}) implementiert.
Diese Mechanismen verhindern, dass der Branch Predictor zwischen logischen Prozessoren der CPU oder Prozessen unterschiedlicher Privilegien geteilt wird.
Dass diese Gegenmaßnahmen nicht immer ausreichen, zeigt Training Solo~\cite{training_solo}, eine Technik, die diese Schutzmaßnahmen umgeht, indem das Trainieren des BTB direkt im Kernel erfolgt.
Somit sind Schutzmaßnahmen wie \textit{IBRS} nicht mehr wirksam, da das Training des BTB von einem privilegierten Prozess ausgeführt wird.
Intel hat bereits reagiert und über Microcode-Updates die Instruktion \textit{IBHF}~\cite{darkcrizt_2025} hinzugefügt.